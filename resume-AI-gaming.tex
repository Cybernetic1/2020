%%%%%%%%%%%%%%%%%
% This is an sample CV template created using altacv.cls
% (v1.1.4, 27 July 2018) written by LianTze Lim (liantze@gmail.com). Now compiles with pdfLaTeX, XeLaTeX and LuaLaTeX.
% 
%% It may be distributed and/or modified under the
%% conditions of the LaTeX Project Public License, either version 1.3
%% of this license or (at your option) any later version.
%% The latest version of this license is in
%%    http://www.latex-project.org/lppl.txt
%% and version 1.3 or later is part of all distributions of LaTeX
%% version 2003/12/01 or later.
%%%%%%%%%%%%%%%%

%% If you need to pass whatever options to xcolor
\PassOptionsToPackage{dvipsnames}{xcolor}

%% If you are using \orcid or academicons
%% icons, make sure you have the academicons 
%% option here, and compile with XeLaTeX
%% or LuaLaTeX.
% \documentclass[10pt,a4paper,academicons]{altacv}

%% Use the "normalphoto" option if you want a normal photo instead of cropped to a circle
% \documentclass[10pt,a4paper,normalphoto]{altacv}

\documentclass[10pt,a4paper]{altacv}
%% AltaCV uses the fontawesome and academicon fonts
%% and packages. 
%% See texdoc.net/pkg/fontawecome and http://texdoc.net/pkg/academicons for full list of symbols.
%% 
%% Compile with LuaLaTeX for best results. If you
%% want to use XeLaTeX, you may need to install
%% Academicons.ttf in your operating system's font 
%% folder.


% Change the page layout if you need to
\geometry{left=1cm,right=9cm,marginparwidth=6.8cm,marginparsep=1.2cm,top=1.25cm,bottom=1.25cm,footskip=2\baselineskip}

% Change the font if you want to.

% If using pdflatex:
\usepackage[T1]{fontenc}
\usepackage[utf8]{inputenc}
% \usepackage[default]{lato}

\usepackage[document]{ragged2e}

% If using xelatex or lualatex:
\setmainfont{Alibaba Sans}

\ifdefined\chinchin
\usepackage[CJKspace]{xeCJK}
% \setCJKmainfont[BoldFont=AlibabaSans-Light,ItalicFont=AlibabaSans-Light]{AlibabaSans-Light}
\setCJKmainfont{Alibaba PuHuiTi}
\newcommand{\cc}[2]{#1}
\else
\newcommand{\cc}[2]{#2}
\fi

% Change the colours if you want to
\definecolor{LightBlue}{HTML}{883333}
\definecolor{LightGrey}{HTML}{333333}
\colorlet{heading}{LightBlue}
\colorlet{accent}{LightBlue}
\colorlet{emphasis}{LightGrey}
\colorlet{body}{black}

% Change the bullets for itemize and rating marker
% for \cvskill if you want to
\renewcommand{\itemmarker}{{\small\textbullet}}
\renewcommand{\ratingmarker}{\faCircle}
%% sample.bib contains your publications
\addbibresource{sample.bib}

\usepackage[colorlinks]{hyperref}

\begin{document}

\name{甄景贤 YKY (Yan King Yin)}
\tagline{獨立 人工智能 研究者}
\photo{2.8cm}{John_Grothendieck.png}
\personalinfo{%
  % Not all of these are required!
  % You can add your own with \printinfo{symbol}{detail}
  \location{香港}    \mailaddress{愉景湾 观景楼 7B} \\
  \email{general.intelligence@gmail.com } \\
  \phone{(+86) 1314-755-0504}   \phone{(+852) 6814-5504}
  % \twitter{@NISZhU}
  % \linkedin{linkedin.com/in/ulbke-zhanysbek-a17459177/}
  % \github{github.com/Cybernetic1}
  %% You MUST add the academicons option to \documentclass, then compile with LuaLaTeX or XeLaTeX, if you want to use \orcid or other academicons commands.
%   \orcid{orcid.org/0000-0000-0000-0000}
}

%% Make the header extend all the way to the right, if you want. 
\begin{fullwidth}
\makecvheader
\end{fullwidth}

%% Depending on your tastes, you may want to make fonts of itemize environments slightly smaller
% \AtBeginEnvironment{itemize}{\small}


%% Provide the file name containing the sidebar contents as an optional parameter to \cvsection.
%% You can always just use \marginpar{...} if you do
%% not need to align the top of the contents to any
%% \cvsection title in the "main" bar.
\cvsection[page1sidebar]{简介}

% \cvevent{Front-End Developer Assistant}{}{}{}

\begin{itemize}
\item 我大部份知识是自学而来,加上和学术界交流,还有网上爱好者的帮助
\item 从小学开始已懂编程,当时用 TRS-80(世界上第一部私人电脑),在香港中大时主修电脑
\item 在美国留学时对 transhumanism 产生兴趣,故自 2004 年起,开始全力研究 普适人工智能
\item 我很喜欢以深入浅出的方式教人,所谓「教学相长」,教导别人时自己也有很大的得益
\item 近年我主要进修高等数学
\item 可以参考我最新的一些 presentation slides
\item 目前有 43 个项目在 Github 上: \\
	{\footnotesize https://github.com/Cybernetic1}
\item 在 知乎 上有不少人工智能方面的文章: \\
	{\footnotesize https://www.zhihu.com/people/Cybernetic1}
\item 我有兴趣开发 gaming environment,可以让 AI 学习一些 physical 或语言上的人类知识
\end{itemize}
谢谢您们的考虑!

% \divider
\medskip

\cvsection{研究兴趣}

\begin{itemize}
	\item 普适 人工智能 (artificial general intelligence, AGI)
	\item 人工智能逻辑, logic-based AI
	\item 自然语言理解, common-sense reasoning
	\item 神经网络、深度学习
	\item (深度)强化学习
	\item 进化算法
	\item 认知架构,认知神经科学/模拟大脑智能
\end{itemize}

\medskip

\cvsection{论文}

\begin{itemize}
	\item Fuzzy-probabilistic logic for common sense \\
	(AGI 2012 Conference at Oxford UK)
\end{itemize}

%\cvsection{A Day of My Life}
%
%% Adapted from @Jake's answer from http://tex.stackexchange.com/a/82729/226
%% \wheelchart{outer radius}{inner radius}{
%% comma-separated list of value/text width/color/detail}
%\wheelchart{1.5cm}{0.5cm}{%
%  6/8em/accent!30/{Sleep and eat}, 
%  4/8em/accent!8/{Socializing},
%  3/7em/accent!8/Go for a walk with friends,
%  8/8em/accent!55/Make my to do list,
%  2/10em/accent!10/Sports and relaxation,
%  5/6em/accent!20/Spending time with family
%}

\medskip

\cvsection{著作/科普文章}

\begin{itemize}
	\item 什么是 机器学习?
	\item 什么是 贝叶斯网络?
	\item 什么是 神经网路?
	\item 什么是 强化学习?
	\item 什么是 遗传/进化算法?
	\item 《计算范畴论》导论
	\item 范畴逻辑简介
	\item 《强人工智能导论》(初稿)
	\item AGI 的一些基本概念
	\item 论 AGI 架构
\end{itemize}

\clearpage


%% If the NEXT page doesn't start with a \cvsection but you'd
%% still like to add a sidebar, then use this command on THIS
%% page to add it. The optional argument lets you pull up the 
%% sidebar a bit so that it looks aligned with the top of the
%% main column.
% \addnextpagesidebar[-1ex]{page3sidebar}

\end{document}
